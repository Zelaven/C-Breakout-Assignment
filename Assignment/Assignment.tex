
\documentclass[12pt]{article}

\begin{document}

% Titlepage
\begin{titlepage}
\center
\textsc{\huge Breakout in C with SDL2}
\rule{\linewidth}{0.5mm}
\\[1em]
\textsc{\large Patrick Jakobsen}
\end{titlepage}

% Table of Contents
\pagenumbering{gobble}
\tableofcontents

% Settings for remainder of the document
\newpage
\pagenumbering{arabic}

% Document contents

\section{Introduction}
This project is for people who already know a little C programming to train their skill with the language. Breakout is a relatively simple game with primarily visually debuggable problems, such that playing the game for a brief time is a reliable way to test if a change functions as intended.\\
Besides the C programming language itself, this assignment uses SDL2, which is a multimedia library, to handle all multimedia functionalities. The core tasks only need it for creating a window and drawing into it, and to handle user input, but it leaves open the possibility to work with additional aspects such as sound. Another advantage to using SDL2 is that it is cross-playform, and this assignment can hence be done on all common operating systems.

\subsection{How to Do this Assignment}
The assignment is compartmentalized, such that is a given part is troublesome, or is done over multiple sessions with a group (The original purpose of this assignment) and attending for one of the parts isn't possible, then that segregated part of the assignment can be taken from the 'Example' folder and inserted into one's own implementation, after which it should work if its dependencies function correctly.\\
\\
The assignment is set up such that I, the author, think that the difficulty progression is ideal, after having taken dependencies within the project into account, and doing the tasks in a linear fashion as of this document is at a bare minimum workable.\\
\\
There are two kinds of tasks. The "mandatory" tasks, which means that without completing them you haven't made a breakout game or are lacking underlying framework that the game needs to function, and the optional tasks, which are additional features that can be added into the game, such as powerups, sound or keeping levels stored in files.\\
\\
Other than that, simply continue reading this document. After you finish reading about the first task and you should be ready to start working on it.\\
As a final note, it is only recommended to look in the Example folder if you are completely stuck and you already have asked people about your problem. It's much better to learn from experimentation and talking with people than to just look at the solution to a problem, and you can always look at how it is solved in the example after you can solved it yourself! That way you get the best of both worlds.\\
\\
\\
I hope you will have a good time with this assignment. Thank your for spending your time on it.

\section{Breakout}
If you don't know what breakout is, then simply use a search engine to look it up. It is likely that you will be able to find a browser-runnable implementation of it that you can play to get a feel for the game.

\section{}



\end{document}
